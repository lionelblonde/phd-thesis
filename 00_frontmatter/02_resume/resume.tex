%!TEX root = ../../main.tex

% resume

\thispagestyle{empty}
\chapter*{Résumé}
% \addcontentsline{toc}{chapter}{Résumé}

Les travaux de recherche décris dans cette thèse visent les faiblesses majeures
de l'apprentissage par renforcement; branche de l'apprentissage automatique
qui tâche de concevoir des agents interactifs autonomes capables de choisir la
sequence de decisions accompagnée des plus fortes récompenses.
De tels agents dependent donc fortement de la façon dont laquelle ces récompenses sont formulées.
Concevoir des récompenses parvenant à guider l'agent vers la résolution de la tâche à laquelle
il fait face est laborieux et requiert de nombreuses interventions et réajustements.
Ces derniers s'avèrent souvent onéreux et dangeureux en raison de la nature intéractive de
l'apprentissage de l'agent qui apprend en intéragissant avec son environment.
L'apprentissage par imitation n'est en revanche pas entravé par l'assemblage minutieux
d'un signal de récompense, qui pourrait même ne pas être \textit{in fine} aligné avec la
tâche que l'agent doit résoudre.
Au lieu de l'accumuler un maximum de récompenses lors de ses interractions avec son
environment, l'agent doit imiter le comportement d'un expert,
qui est supposé \textit{a priori} optimal pour la tâche à accomplir, et
qui lui est communiqué par l'intermediaire de démonstrations.
Même si ces méthodes évitent l'étape préliminaire consistant à aligner un signal
de récompense avec la tâche à résoudre, les meilleures methodes d'apprentissage
par imitation demeuraient couteuse en terme du nombre d'interactions que l'agent
devait effectuer avec l'environment avant d'espérer adopter un comportement proche
de celui de l'expert.

Ainsi, nous cherchons dans un premier temps
à reduire le nombre d'échantillons dont la méthode de pointe,
l'apprentissage par imitation antagoniste generative,
a besoin pour imiter l'expert.
Pour ce faire, nous rendons la méthode ``off-policy'',
ce qui permet à l'agent d'extraire davantage d'information de ses interactions passées
en les préservant en mémoire et s'en remémorant occasionnelement.
Malgré l'instabilité notoire des méthodes impliquées
--- apprentissage ``off-policy'' et methodes antagonistes generatives ---
notre nouvelle méthode est quant à elle remarquablement stable,
et reduit le nombre d'interactions nécessaires d'au moins un ordre de grandeur.

Afin d'élucider pour quelles exactes raisons la méthode d'imitation
dévelopée reste si stable en dépit des fortes chances de divergence,
nous étudions la méthode proposée en profondeur.
Ainsi, dans un deuxième temps, nous démontrons que,
afin d'assurer la stabilité de la méthode,
la fonction de récompense apprise de façon antagoniste doit
être une application lipschitzienne.
Nous étudions les effets de cette condition nécessaire et dérivons plusieurs
résultats théoriques caractérisant la fonction de valeur état-action,
communément notée ``Q''.
Ces garanties sont complémentées de preuves empiriques attestant de l'importance
qu'ont les contraintes encourageant la récompense apprise
à une application lipschitzienne
sur la performance de l'agent.
Enfin, nous focalisons notre attention sur une large classe de préconditioneurs
visant à être appliqués à la récompense, et montrons via plusieurs garanties théoriques
que les membres de la classe introduite rendent les récompenses auxquelles ils sont
appliqués plus robuste.
Les garanties dérivées s'étendent à toutes les récompenses lipschitziennes;
elles ne se limitent pas à l'apprentissage par imitation.

Imiter le comportement d'un tier agent ne fait sens que lorsque ce dernier agis de
façon optimale, mais l'obtention de démonstrations d'un tel expert est loin d'être
une mince affaire, et devient vite laborieux lorsque les objectifs s'avèrent complexes.
Des traces d'interaction décrites par un comportement sous-optimal sont bien plus
facile à obtenir (\textit{e.g.}~capteurs d'une voiture autonome prenant continuellement
des mesures lors de chaque navigation).
Ainsi, après avoir étudié l'apprentissage de comportement interactif
à partir de données provenant d'un expert supposé optimal,
nous considérons, dans un troisième temps,
la situation dans laquelle ces données de source(s) extérieure(s)
sont potentiellement sous-optimales,
et la traitons par apprentissage par renforcement hors-ligne (ou ``offline'').
Alors que l'agent qui apprenait par imitation ne recevait pas de récompense
de son environement, il était tout du moins autorisé à interagir avec ce dernier.
Pour l'agent qui apprend par renforcement hors-ligne, la situation est inversée.
L'agent a accès a un historique d'experiences correspondant aux intéractions passées
d'un ou de plusieurs autres agents, où chaque situation et décision prise
par le tiers agent en question
est annoté de la récompense qu'il a perçu lorsque la décision a été exécutée.
En revanche, étant hors-ligne, l'agent ne peut lui-même pas interagir avec son environment,
et ne peut de ce fait pas observer l'effet de ses propres décisions alors qu'il essaie de
s'améliorer.

La performance d'agents apprenant par renforcement hors-ligne fluctue grandement en fonction
de la qualité des données, qui peut varier
sur une échelle allant de
``loin d'être optimal'' (\textit{e.g.} aléatoires)
à ``près d'être optimal''.
En nous reposant sur notre ré-implémentation des méthodes de pointe d'apprentissage par
renforcement hors-ligne au sein d'un système unifié,
nous montrons que lorsqu'une de ces méthodes de pointe surpasse ses concurrentes
sur une extremité de l'echelle, elle ne les surpasse \emph{jamais} sur l'autre extremité.
Ce motif récurrent nous empêche de nommer un vainqueur sur l'intégralité de l'échelle de qualité.
Nous attribuons cette asymétrie au biais inductif qu'injecte ces méthodes dans leur agents
réspectifs, où le biais en question pousse l'agent à partir du principe que le
comportement décris dans les données hors-ligne est \emph{optimal} pour la tâche
à effectuer.
Plus il y a de biais injecté, meilleur sera l'agent \emph{à condition que}
le dataset hors-ligne soit ``près d'être optimal''.
Sinon, l'injection a un effet néfaste.
Nous les nommons biais inductif d'optimalité.
En nous fondant sur une méthode de régression pondéré par la function valeur état-action,
nous menons une investigation qui corrobore que l'injection de bias inductif d'optimalité
reduit considérablement la performance de l'agent sur des données sont sous-optimales,
si cette injection n'est pas effectuée avec parcimonie.
Afin de concevoir une classe de méthodes qui permettrait à l'agent d'avoir de bonnes
performance sur l'intégralité de l'echelle de qualité,
nous revisitons le schéma de l'itération de politique généralisée.
Nous étudions l'impact de nouvelles distributions sur l'espace de décisions
dans le cadre d'une généralisation spécifique au régime hors-ligne
de ce schéma que nous introduisons.
Nos études montrent qu'il existent des orchestrations de distributions dans la
généralisation proposée qui permettent à l'agent d'aboutir à un bon équilibre
de performance où l'agent devient bien plus
compétitif sur l'intégralité de l'échelle de qualité de données.
