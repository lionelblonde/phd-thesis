%!TEX root = ../../main.tex

% abstract

\thispagestyle{empty}
\chapter*{Abstract}
% \addcontentsline{toc}{chapter}{Abstract}

The research endeavours laid out in this thesis set out to
address weaknesses of reinforcement learning.
As the branch of machine learning that orchestrates a
reward-driven decision maker that must learn from its evolving world
via interaction, reinforcement learning relies heavily on said reward's design.
Modelling a reinforcement signal able to convey the right incentive to the agent
for it to carry out the task at hand is far from obvious, and overall fairly tedious
in terms of engineering and tweaking required.
Besides, the interactive nature of the domain puts the agent and its world at satefy risk.
The longer it takes to iterate over and empirically validate reward designs,
the less sensible the suffered costs gradually become.
Imitation learning bypasses this time-consuming hurdle by enticing the agent to mimic an expert
instead of trying to maximize a potentially ill-designed reward signal.
In effect, the shift from reinforcement to imitation learning trades the reward feedback the
agent receives upon interaction off for the access to
a dataset describing how an expert decision-making policy
--- assumed to perform optimally at the task --- maps situations to decisions.
Depite alleviating the reliance on potentially satefty-critical reward modelling,
state-of-the-art imitation learning agents still suffered from severe sample-inefficiency in the
number of interactions that need be carried out in the environment.

\emph{%
As such, we first set out to address the high sample complexity
suffered by Generative Adversarial Imitation Learning, the state-of-the-art approach
to performing imitation (albeit sample-inefficient).
To that end, we build on the strengths of off-policy learning
to design a novel adversarial imitation approach.
Albeit coordinating infamously unstable routines --- adversarial learning, and off-policy learning ---
our novel method remains remarkably stable, and dramatically shrinks the amount of interactions with the
environment necessary to learn well-behaved imitation policies,
by up to several orders of magnitude.
}

\emph{%
In an attempt to decipher and ultimately understand what made
the previous method so stable in practice despite overwhelming odds of instability, we
perform an in-depth review, qualitative and quantitative, of the method.
We show that forcing the adversarially learned reward function to be local Lipschitz-continuous
is a \textit{sine qua non} condition for the method to perform well.
We then study the effects of this necessary condition and provide several theoretical results
involving the local Lipschitzness of the state-value function.
We complement these guarantees with empirical evidence attesting to the strong
positive effect that the consistent satisfaction of the Lipschitzness constraint on the reward has
on imitation performance.
Finally, we tackle a generic pessimistic reward preconditioning add-on
spawning a large class of reward shaping methods, which
makes the base method it is plugged into provably more robust, as shown in several
additional theoretical guarantees.
We then discuss these through a fine-grained lens and share our insights.
Crucially, the derived guarantees are valid for \emph{any} reward
satisfying the Lipschitzness condition, nothing is specific to imitation.
As such, these may be of independent interest.
}

While imitation learning seems difficult to outperform
once the agent has access to past experiences from an expert behaving optimally,
the performance of the mimicking agent quickly deteriorates as the expert stray from optimality.
Intuitively, it stands to reason that the agent should only imitate another if the latter
solves the task (near-) optimally. If not, then imitation is not a well suited approach.

Datasets of optimal interactive experiences are difficult to come by,
and deterringly tedious to design from scratch.
Even then, these rarely perfectly align with the practitioner's desiderata for the set task.
As a relaxation, interaction traces of sub-optimal behaviors for closely related (yet not exactly aligned) tasks
are far more readily available
(\textit{e.g.}~due to continually recording sensors present in robots, cars).
As such, after studying how to \emph{only learn from an external source} of \emph{putatively optimal} data coming
from an expert policy, we then study how to \emph{only learn from an external source} of
\emph{putatively sub-optimal} data provided through a arbitrary dataset ---
this last one is the \emph{offline} reinforcement learning setting.

While in the imitation learning setting we had no reward from the environment,
we were still able to interact with the environment.
In the offline reinforcement learning setting, the situation is reversed.
The agent has access to logged interactive experiences from external sources, typically distinct agents.
Said dataset of interactive traces is annotated with rewards perceived by the agents upon interaction.
Yet, the agent learned via offline reinforcement learning
is not allowed to interact with the world, and is therefore unable to witness its \emph{own} impact on it when trying
to make better decisions in it.

\emph{%
The performance of state-of-the-art baselines in the offline reinforcement learning regime varies widely over the
spectrum of dataset qualities --- ranging from \textit{``far-from-optimal''} random data
to \textit{``close-to-optimal''} expert demonstrations.
We re-implement these under a fair, unified, and highly factorized framework,
and show that when a given baseline outperforms its competing counterparts
on one end of the spectrum, it \emph{never} does on the other end.
This consistent trend prevents us from naming a victor that outperforms the rest across the board.
We attribute the asymmetry in performance between the two ends of the quality spectrum to
the amount of inductive bias injected into the agent to entice it to posit that what the behavior underlying
the offline dataset is \emph{optimal} for the task.
The more bias is injected, the higher the agent performs, \emph{provided the dataset is close-to-optimal}.
Otherwise, its effect is brutally detrimental.
Adopting an advantage-weighted regression template as base, we conduct an investigation which corroborates that
injections of such \emph{optimality} inductive bias, when not done parsimoniously,
makes the agent subpar in the datasets it was dominant as soon as the offline policy is sub-optimal.
To design methods that perform well across the whole spectrum,
we revisit the generalized policy iteration scheme for the offline regime, and study the impact of nine distinct
newly-introduced proposal distributions over actions, involved in the proposed generalization of the
policy evaluation and policy improvement update rules.
We show that certain orchestrations strike the right balance and can improve the performance on one end of the spectrum
\emph{without} harming it on the other end.
}

In neither of the two studied settings
(\textit{a)} imitation learning and \textit{b)} offline reinforcement learning)
does the agent receive direct reward feedback from the world
about the quality of the decisions it makes.
It is never told \emph{how well} it performed, and must infer how well it performed from
traces of other policies, which \textit{a priori} do not align with what the decision maker would have done.
In that light, both studied settings
can be cast as counterfactual learning,
in which sequential decision-making agents must learn how to interact with their world,
only from traces that \textit{a priori} do not coincide with their own.
\textit{In fine}, we are concerned with decision makers that learn from the choices of others.


