%!TEX root = ../../main.tex

% final

\chapter[Final thoughts]{Final thoughts}
\label{thesis:final}

We have studied two ways of teaching an artificial agent to learn to interact with a world from
which it does not get direct feedback on its evolving decision process.
In imitation learning, the decision maker receives demonstrations from an expert, assumed to behave optimally;
it can interact with its environment but does not receive the reward response upon execution.
The decision-maker does not know how its own behavior would have been rated by the environment throughout the
learning process, and assumes that the demonstrations it is provided with are what said environment would
rate as optimal. We have treated this setting in \textsc{Chapters}~\ref{thesis:chap1} and \ref{thesis:chap2}.
Similarly, when reinforcement learning is brought offline, the decision maker is never explicitly
told how well it does at completing the interactive task because it is never allowed to interact with its
environment in the first place.
Instead of being rated directly about its own behavior, it is made aware of
how another agent was rated by the environment when attempting to complete the same task.
Note, in imitation learning, the demonstrations are not annotated with rewards like in offline RL,
hence leaving the decision maker with the sole choice of having to interpret the demonstrations as optimal.
In offline RL however, the provided traces of interactions from another agent are rated with rewards,
and as such, the content of the the provided offline dataset need not be interpreted as optimal.
In \textsc{Chapters}~\ref{thesis:chap3}, we have treated this setting and
shown that seeing the datasets as optimal
(despite being annotated with rewards that would enable a more agnostic approach with respect to the
offline dataset's optimality) leads to poor performance compared to agnostic approaches
when the dataset contains sub-optimal data.

Learning an interactive strategy without direct feedback on the decisions made,
a setting we identified as an instance of counterfactual learning,
has proven challenging, and is by no means a solved problem.
As soon as the decision maker has limited access to the environment (be it the lack of direct reward feedback
upon interaction, or the inability to interact altogether)
its ability to learn to complete an interactive task in the world is often significantly impaired.
Leveraging datasets of logged interactions captured from another agent exposes the agents to
hindering stability issues due to distributional mismatches occurring during the learning process.
Still, considering the ubiquity and mass production of various types of sensors,
aggregating data into feature-rich datasets of logged interactions is convenient and cheap.
It is also far safer than the online (artificial counterpart of \textit{``in-person''}), interactive
alternative: learning by acting in the real world.
The exploration involved in reinforcement learning quickly makes the entire learning process hazardous
both for the artificial decision maker (\textit{e.g.} autonomous vehicle, robot)
and its environment (\textit{e.g.} pedestrians, healthcare patients).
Looking for a digital surrogate to real-world interactions,
the RL practitioner has naturally turned to training decision maker in simulation.
Simulators offer valuable flexibility when it comes to exposing the agent to a wide and diverse
array of situations that the agent would only rarely be faced by in the real world
(\textit{e.g} unusual pavement coloration, signs being held by road workers in the occurrence of a road collision).

High-fidelity, photo-realistic simulators that would best prepare the agent to face the real world,
are nevertheless incredibly difficult and finicky to craft.
Besides, they are usually used in tandem with augmented datasets of logged interactions from
a multitude of agents.
Challenges of such combination of simulation and (augmented) real-world data include:
\emph{can the crafted simulator represent the data captured from the real world in simulation
without mismatch?}
\emph{How should the reward function underlying the observed behavior be designed
to entice an agent learned in simulation to reproduce the same behavior?}
\emph{How does one enable the artificial agent to deal with hazardous situations if
everything that has ever being recorded corresponds to safe behaviors?}
\emph{How should one bake in the multi-agent and social aspect of real-world interactions in simulation?}
The social aspect of most real-world scenarios (in which RL seems to be the right paradigm to teach an AI)
is particularly tedious to deal with considering how abstract concepts such as self-awareness enabling the agent
to recognize its counterparts in the world can not be easily inferred from sensory recordings.
By being able to identify an entity moving in its environment as another learning agent, like itself,
the agent would be able to forecast what this external agent would to if it were to follow the main agent's
decision process. By being able to forecast how such multi-agent environment might evolve
in the next few seconds or minutes, the agent is likely to make better decisions overall.
As such, designing social agents able to predict how other actors might act and react to
any of the other actors' decisions is an important line of research to pursue,
which can take advantage of the relational learning capabilities of recent attention-based research advances.
Examples of such endeavors include
\cite{Bradley_Knox2013-jg,Alahi2016-vi,Leibo2017-lg,Tai2017-wq,Vemula2017-zd,
Li2018-wc,Gupta2018-cu,Kosaraju2019-ui,Zhou2020-hs,Ndousse2020-oz},
where self-driving is focused most.

We can expert high-fidelity simulators becomes better and better as times goes on,
closing the gap between learning in simulation and real-world interaction.
We can notably expect these to encode real-world physics with an increasing accuracy,
to the point where the physics might even be baked into the decision maker itself.
An example of the exploitation of such physics-originated properties is the explicit use
of clothoids in \cite{Zeng2020-co}, the curves described by vehicles when turning the steering wheel at
constant angular speed.
These curves are instrumental in high-speed road design to join roads where the steering wheel is locked
at a zero angle (the trajectory describes a straight line) and roads where the steering wheel is locked at a
non-zero angle (the trajectory describes a circle).
Clothoids are used because they translate to the most comfortable transition between these two
locked steering wheel positions for the driver.
They are also known as Euler spirals, or Cornu spirals is optics.

As we close the gap between simulated and real-world reinforcement learning, it will prove increasingly valuable to
adopt a multi-agent lens and provide the agent (or agents) with mechanisms
that enable it (or them) to display social traits and manifest relational reasoning.
Decentralized multi-agent RL, where every involved decision maker is independent,
has the best horizontal scalability (compared to its centralized counterpart), but is also
hindered by considerable extra challenge:
from any of the independent agent's perspective, the environment (containing the other independent agents) is
constantly changing.
The non-stationarity of the respective environments every independent agent interacts with
makes the task considerably more tedious to tackle empirically and violates the Markov property
on which most convergence guarantees rely.
It is common to base the agent's decisions on a history of states (or on a summary thereof) in such scenarios,
to palliate the non-stationarity of the environment's dynamics from any agent's point of view.
Sequential or memory-augmented models might provide the agent with the ability to infer the other agents' intents
when the current state is not a sufficient statistic of the future.
We can therefore expect progress in general sequential modeling to assist progress in
these challenging multi-agent settings.
As the simulated environments adopt structures that are increasingly more truthful in complexity
to the real world, which is inherently non-stationarity,
research progress on transferring models learned in simulation in the real world (\textit{``sim-to-real''}) will
without a doubt play a significant role in closing the simulation-reality gap over time.

Still, once we succeed in transferring agents trained in simulation into the real world,
leveraging both captured data and high-fidelity simulators,
the taught artificial agent must be adopted by the human user and be appealing to interact with.
Humans effectively are agents in the multi-agent system, and as such,
aspects from human-computer interfaces must be considered to integrate learned RL agent in
the daily activities of the human end users.
Imitation learning can play a significant role in making interactions with artificial agents
more natural and human-like, as given an account of in
\cite{Livingstone2006-ln,Tence2010-qt},
tackling \emph{believability} in the context of AI in games.
Both conversational and embodied agents can be made more believable, human-like,
and therefore seemingly trustworthy by enabling the artificial decision maker to manifest
interpersonal \emph{synchrony}
(rhythmic coordination of actions, emotions, speech)
with its human user,
as synchrony plays an essential role in the way humans bond.



